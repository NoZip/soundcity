\section{Génération}
\label{impl:generation}

L'objectif du générateur étant d'assurer une similarité entre tous les
morceaux de la playlist, mais surtout d'assurer la similarité des pistes deux
à deux dans l'enchaînement des morceaux, il a été choisi d'implémenter notre
algorithme de génération sous forme itérative.

\subsection{Algorithme naïf}
\label{impl:selection:naif}

La première solution proposée pour implémenter la génération de playlist a été
de réaliser un algorithme glouton et naïf.\newline

L'algorithme est le suivant~:

\begin{enumerate}
  \item Début.
  \item Le premier morceau de la Pool devient le premier morceau de la Playlist.
  \item Tant que la Playlist n'a pas la taille requise et que la Pool n'est pas
vide~:
  \begin{enumerate}
    \item On met le maximum de smilarité à 0.
    \item Pour chaque morceau t dans la Pool~:
    \begin{enumerate}
      \item On calcule le score de similarité entre t et le dernier morceau
      de la Playlist.
      \item Si le score de similarité est supérieur au score maximum~:
      \begin{enumerate}
        \item On remplace le score maximum par le nouveau score maximum.
        \item On enregistre le morceau qui a le meilleur score de similarité.
      \end{enumerate}
    \end{enumerate}
    \item On ajoute le morceau qui a la similarité maximale à la fin de la
    Playlist actuelle.
  \end{enumerate}
  \item Si la similarité globale de la Playlist est suffisante~:
  \begin{enumerate}
    \item On marque la Playlist comme valide.
  \end{enumerate}
  \item Fin.
\end{enumerate}

Les résultats de cet algorithme sont bons, mais il n'est cependant pas optimal.
En effet, à chaque ajout de morceau dans la Playlist, cet algorithme ne vérifie
pas la similarité du morceau testé avec tous les morceaux de la Playlist, et
n'assure pas que le morceau après lequel il est ajouté est celui avec lequel
il a le plus grand score de similarité.


\subsection{Algorithme optimal}
\label{impl:selection:optimal}

Comme énoncé précedemment, le premier algorithme implémenté n'est pas optimal.
Il a donc été important de chercher un algorithme proposant une solution
optimale, ou capable de s'en approcher.\newline

La solution trouvée est de considérer tous les morceaux de la Pool, et d'en
constituer un graphe complet reliant tous les morceaux les uns aux autres. Les
arêtes seraient alors pondérées par le score de similarité entre les morceaux
situés à leurs extrémités. Une fois le graphe construit, il faut appliquer un
algorithme capable de calculer le coût de tous les chemins de taille N (avec N
la taille souhaitée pour la Playlist).\newline

Un nouveau problème se pose alors~: le calcul de similarité utilise les
informations contenues dans la base de données. Ne sachant pas la précision de
ces informations, la question est alors de savoir s'il est pertinent
d'utiliser un algorithme de grande complexité et de chercher un résultat
optimal à partir de données non vérifiables.

\subsection{Optimisation}
\label{impl:selection:optimisation}

Suite à la question que pose l'utilisation d'un algorithme optimal, il a été
décidé que celui-ci ne serait pas implémenté, et qu'il serait préférable
d'améliorer le résultat obtenu par l'algorithme naïf en y apportant une légère
optimisation.\newline

En effet, la faiblesse de cet algorithme est de ne comparer les morceaux
restants dans la Pool qu'au dernier morceau ajouté dans la Playlist. Cette
dernière étant implémentée par une liste doublement chaînée, il a alors été
décidé de comparer les éléments restants dans la Pool au dernier, mais aussi au
premier élément de la Playlist afin d'augmenter les chances d'obtenir un
meilleur résultat.


\section{Similarité}
\label{impl:similarite}

Afin de conserver une cohérence avec la façon dont sont représentées les
différentes données autour des morceaux, il a été choisi de toujours ramener
le score de similarité calculé entre 0 et 1.

\subsection{Pattern strategy}
\label{impl:similatite:strategy}

Comme énoncé précédemment (c.f. Section
\ref{besoins:nfonc:perf:mod:similarity}), il a été choisi d'implémenter le 
calcul de similarité en suivant le pattern strategy.\newline

Ce choix a été guidé par le fait que la base de données utilisée par notre
générateur peut être changée par l'utilisateur. Il faut donc pouvoir
facilement adapter notre calcul de similarité à une nouvelle base de données
afin d'assurer que le générateur conserve sa pertinence.

\subsection{Calcul de similarité naïf}
\label{impl:similarite:naif}

Dans un premier temps, nous avons choisi d'implémenter un calcul de similarité
dit «~naïf~» afin d'assurer un fonctionnement minimal à notre générateur. 
Le but de ce calcul n'est pas d'être pertinent, mais de permettre de valider
l'ordonnancement de plusieurs morceaux en fonction d'un seul critère.

Pour cette implémentation du calcul, on se base uniquement sur la différence
entre les années de sortie des albums dont sont extraits les morceaux, à
laquelle est appliquée un logarithme décimal. L'utilisation du logarithme
décimal sert à considérer l'écart d'années entre les morceaux de façon
non-linéaire afin d'atténuer les variations trop importantes. Le calcul est
donc le suivant (avec date1 et date2 les années de sortie respectives des 
albums des pistes track1 et track2)~:

\begin{equation*}
  score(track1, track2) = log_{10}(date1 - date2)
\end{equation*}\newline

Enfin, pour ramener le score de similarité entre 0 et 1, on applique les
opérations suivantes en considérant un seuil en dessous duquel la différence
d'années est considérée négligeable~:

\begin{itemize}
\item score = 1, si le score est supérieur à 1. 
\item score = 0, si le score le score est inférieur au seuil.
\item On retourne (1 - score) afin que 0 représente une similarité nulle et 1
une similarité totale.
\end{itemize}

\subsection{Calcul de similarité complet}
\label{impl:similarite:complet}

Dans l'implémentation de ce calcul de similarité dit «~complet~», la formule
utilisée pour produire un score prend en compte plusieurs caractéristiques des
morceaux comparés. Ces caractéristiques sont extraites de la base de données
et sont les suivantes~:

\begin{itemize}
\item Le rythme.
\item L'énergie.
\item La date de sortie de l'album dont est extrait le morceau.
\item Le nom de l'artiste.
\end{itemize}

On peut imaginer la répartition de chaque morceau dans un repère à 4
dimensions où chacun des axes représente une de ces quatre caractéristiques.
Le score de similarité est alors calculé en faisant une moyenne de la
variation entre les deux morceaux sur chacun de ces axes.

Afin d'établir un ordre d'importance entre ces caractéristiques, on décide de
pondérer cette moyenne avec les facteurs suivants~:

\begin{itemize}
\item Le rythme compte pour 40\%.
\item L'énergie compte pour 30\%.
\item La date de sortie de l'album compte pour 20\%.
\item Le nom de l'artiste compte pour 10\%.
\end{itemize}

Ces facteurs de pondération ont été choisis de façon à ce que les données
signal representent 70\% du score de similarité, et les données contextuelles
les 30\% restants.

\subsubsection{Rythme et énergie}
\label{impl:similarite:complet:rhythm}

La variation entre les deux morceaux sur les axes représentant le rythme et
l'énergie se fait de façon classique en inversant le résultat, afin de toujours
considérer 0 comme une similarité nulle et 1 comme une similarité totale comme
suit (avec rhythm1, energy1, rhythm2 et energy2 respectivement les rythmes et
énergies des deux morceaux comparés) ~:

\begin{align}
  \Delta Rhythm = 1 - |rhythm1 - rhythm2|\\
  \Delta Energy = 1 - |energy1 - energy2|
\end{align}

\subsubsection{Date de sortie de l'album}
\label{impl:similarite:complet:date}

La variation entre les dates de sorties des albums dont sont extraits les deux
morceaux est calculée de la même façon que dans le calcul de similarité naïve.
(c.f \ref{impl:similarite:naif})

\subsubsection{Nom d'artiste}
\label{impl:similarite:complet:name}

On considère que deux morceaux du même artiste ont de grandes chances d'être
similaires. Cette carectèristique étant faiblement pondéré dans le calcul de
similarité, on peut définir la variation sur le nom d'artiste comme suit~:

\begin{align}
  \Delta Artist = \text{1, si artist1 = artist2.}\\
  \Delta Artist = \text{0, sinon.}
\end{align}

\section{Module de données SQLite}
\label{implementation:sqlite}

L’implémentation du module de données SQLite (classe \texttt{SQLiteDatabase})
à été réalisée en utilisant la bibliothèque sqlite3
\footnote{http://www.sqlite.org/about.html}. Elle permet une interface avec les
fichiers sqlite. Elle est écrite en C, mais cela n'a pas posé de problème
particulier.

L’implémentation de la sélection dans le module SQLite est divisée en plusieurs
phases~:
\begin{description}
  \item[La création de la requête~:] Au format SQL, qui doit prendre en compte
  les options passsés en paramètre à \texttt{select}. Des buffers de chaîne de
  caractères ont été utilisés, permettant de meilleures performances. L'usage de
  la fonction \texttt{random} avec la directive \texttt{ORDER BY}, permet la
  sélection aléatoire.

  \item[Execution et traitement de la requête~:] La requête est exécutée, puis
  chaque ligne renvoyée par la base de données SQLite est transformé en une
  piste, qui est ajouté à une liste temporaire.

  \item[Ajout des artistes similaires~:] Il faut demander à la base de données
  quels sont les artistes similaires pour chaque piste sélectionnée. Le procédé
  est relativement similaire à la phase précédente, mais est exécutée pour
  chaque piste.

  \item[Copie des pistes dans un \texttt{Trackpool}~:] Pour finir, la liste
  temporaire est copiée dans un \texttt{Trackpool}, qui sera retourné par la
  fonction. La liste temporaire était nécéssaire, car la \texttt{Trackpool} ne
  permet pas de modifier ses éléments.
\end{description}
