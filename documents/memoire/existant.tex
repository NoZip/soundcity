
\section{Qualité d'une playlist}
\label{existant:qualite}

Les conférenciers Ben Fields et Paul Lamere ont présenté une très bonne
conférence \cite{ismir2010:playlist-tutorial} qui constitue une bonne approche
de la construction de playlists.

Ils définissent la playlist comme étant un ensemble de morceaux \emph{ordonnées}
\cite[p.~7]{ismir2010:playlist-tutorial}. L'ordre des morceaux est donc
important.

Ils décrivent une playlist de qualité comme étant influencée par un certain
nombre de facteurs \cite[p.~17--18]{ismir2010:playlist-tutorial}. Le facteur qui
nous intéresse particulièrement est le «~niveau de variété et de cohérence~» de
la playlist.

La \emph{cohérence} est présentée \cite[p.~21 -- 23]{ismir2010:playlist-tutorial}
comme une aide à la construction de mix permettant de rapprocher les morceaux.
Le paramètre de cohérence qui nous intéresse ici est la «~Song Similarity~» qui
définit la similarité entre des morceaux en analysant le signal audio.
Les paramètres «~Artist / Genre / Style~» sont également important, mais 
relèvent du contexte du morceau.

Une autre composante importante est la \emph{variation} 
\cite[p.~32]{ismir2010:playlist-tutorial}. En effet, une cohérence trop forte
dans une playlist provoquera une certaine monotonie. Il faut donc éviter cela
en prenant en compte la variation dans la qualité de la playlist.

\section{Générateurs de playlists existants}
\label{existant:generateurs}

\subsection{Musicovery}
\label{existant:generateurs:musicovery}

Musicovery\footnote{http://musicovery.com/} est un site web qui propose une 
lecture en streaming d'une playlist à partir de morceaux sélectionnés 
selon plusieurs types de paramètres, qui peuvent être combinés afin d'affiner 
la recherche.

En effet, le site propose d'effectuer des recherches par artiste, par tag, 
par année, ou par genre. Mais son originalité réside en un placement des 
morceaux dans un graphique en deux dimensions selon les deux axes suivants:
sombre/positif et énergique/calme.
Ainsi, chaque utilisateur de ce site peut écouter une playlist de morceaux 
variés et d'artistes différents, tout en restant dans le même thème, selon 
son humeur du moment.
En plus de cela, l'utilisateur peut demander une playlist de pistes plus 
ou moins connues et populaires grâce aux deux options «~Hits~» et «~Découverte~».

Le site propose également quelques options supplémentaires comme le partage 
sur les réseaux sociaux, ou encore une inscription afin de garder en mémoire 
des morceaux favoris.

\subsection{Spotibot}
\label{existant:generateurs:spotibot}

Spotibot\footnote{http://www.spotibot.com/} est un site internet permettant de 
créer une playlist en fonction d'un (ou plusieurs) nom d'artiste entré par l'utilisateur,
 ou en se basant sur son profil Last.fm. Comme le nom du 
site peut le laisser penser, la lecture de la playlist ainsi générée est 
effectuée sur Spotify.

Sur ce site, les options et les précisions que peut faire l'utilisateur sont 
assez limitées. En effet, il peut uniquement choisir la taille de la 
playlist demandée (en nombre de morceaux), et si il préfère avoir des 
morceaux populaires ou non.

Une fois que l'utilisateur a validé sa recherche, Spotibot lui renvoie une 
playlist du nombre de morceaux demandé, en fonction de la similarité avec 
l'artiste saisi (via les tags, et via l'historique des écoutes du compte 
Last.fm). Une fois cette playlist générée, l'utilisateur peut décider d'y 
ajouter quelques morceaux supplémentaires à l'aide d'un simple bouton, ou 
alors d'enlever des morceaux qui ne lui conviennent pas.

Une fois finalisée, la playlist peut être exportée pour être lue directement 
sur Spotify par simple Drag'n'Drop.

Ce site possède également un top des groupes/artistes les plus populaires et 
les plus recherchés, mais aussi de ceux les plus supprimés.

\subsection{Sourcetone}
\label{existant:generateurs:sourcetone}

Sourcetone\footnote{http://www.sourcetone.com/} est une application qui permet 
de classer des morceaux d'une bibliothèque musicale en différentes catégories, 
afin avoir des playlists correspondant à l'humeur et/ou à l'activité du moment. 
Par exemple, cela permet d'avoir une playlist entraînante pour faire du sport, 
ou une playlist plus calme pour se reposer le soir.

Sourcetone possède également une fonction radio, que l'utilisateur peut 
paramétrer selon ses envies (musique rythmée ou non, joyeuse ou non, de 
quelle année, etc.).

L'application se base sur les évaluations et les avis de l'utilisateur 
lui-même: elle a besoin d'un retour, dans le but «~d'apprendre~» à mieux le 
connaître, afin de lui proposer des pistes de plus en plus adaptées selon 
son humeur ou son activité.


