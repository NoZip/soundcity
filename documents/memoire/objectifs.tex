
\section{Objectifs}
Comme il est expliqué plus haut, plusieurs générateurs de playlists existent
déjà, et permettent de fournir une sélection de morceaux articulée autour 
d'un même thème et de plusieurs paramètres.
Notre objectif principal est de générer une playlist selon les mêmes types 
de critères, en gardant une cohérence globale, mais en attachant une 
importance particulière à l'enchaînement des morceaux. Ainsi, les morceaux 
de la playlist générée par SoundCity se suivent selon un ordre optimal afin 
d'éviter des transitions brutales entre deux musiques qui correspondent aux 
criètres de recherches rentrés, mais qui sont trop différents dans leur 
structure musicale (tempo trop différent, époque trop lointaine, etc.)

L'originalité et l'apport supplémentaire qu'offre SoundCity par rapport aux 
autres générateurs est donc l'enchaînement des morceaux. La partie centrale 
du logiciel est celle qui s'occupe d'ordonner les musiques sélectionnées 
dans la base de données. En effet, la sélection des morceaux correspondants 
aux données rentrées par l'utilisateur n'est qu'un filtrage, tandis que 
l'ordonnancement se concentre sur la similarité de ces musiques afin de les 
faire s'enchaîner au mieux.

La base de donnée utilisée pour le logiciel est tirée de la Million Song 
Dataset (MSD). Comme son nom l'indique, c'est une base de données d'un 
million de morceaux musicaux, qui contient toutes les informations 
nécessaires sur les musiques, et dont la fiabilité est assurée.

Un autre point important que le logiciel doit avoir, c'est le retour 
utilisateur (feedback). La génération de la playlist peut prendre jusqu'à 
plusieurs dizaines de secondes, et pendant ce temps là, il ne faut pas 
laisser l'utilisateur sans aucune information. Pour ce faire, à tout moment 
de la génération, une simple information sur le nombre de morceaux ajoutés à 
la playlist à cet instant est affichée à l'utilisateur.