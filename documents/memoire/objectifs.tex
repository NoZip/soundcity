
\section{Objectif principal}
\label{objectifs:principal}

Comme il est expliqué précédemment (c.f. Section \ref{existant:generateurs}), 
plusieurs générateurs de playlists existent déjà, et permettent de fournir une 
sélection de morceaux articulée autour d'un même thème et de plusieurs 
paramètres.\\

Notre objectif principal est de générer une playlist selon les mêmes types 
de critères, en gardant une cohérence globale, mais en attachant une 
importance particulière à l'enchaînement des morceaux. Ainsi, les morceaux 
de la playlist générée par SoundCity doivent se suivre dans un ordre permettant 
d'éviter des transitions brutales entre deux morceaux qui correspondent aux 
criètres de recherches rentrés, mais qui sont trop différents dans leur 
structure musicale (tempo trop différent, époque trop lointaine, etc.).\\

Il est aussi important que le logiciel ait un retour 
utilisateur (feedback). La génération de la playlist peut prendre plusieurs 
secodnes, et pendant ce temps là, il ne faut pas 
laisser l'utilisateur sans aucune information. Pour ce faire, à tout moment 
de la génération, une simple information sur le nombre de morceaux ajoutés à 
la playlist à cet instant est affichée à l'utilisateur.\\

Pour effectuer la génération de la playlist, l'utilisateur a la possibilité 
d'entrer des paramètres afin d'influencer et de préciser la création de la 
suite de morceaux. Il pourra ainsi sélectionner une tranche d'années s'il le 
souhaite, ou encore un artiste autour duquel articuler l'élaboration de la 
playlist.


\section{Particularités}
\label{objectifs:particularites}

L'originalité et l'apport supplémentaire qu'offre SoundCity par rapport aux 
autres générateurs est donc l'enchaînement des morceaux. La partie centrale 
du logiciel est celle qui s'occupe d'ordonner les pistes sélectionnées 
dans la base de données. En effet, la sélection des morceaux correspondants 
aux données rentrées par l'utilisateur n'est qu'un filtrage, tandis que 
l'ordonnancement se concentre sur la similarité de ces morceaux afin de les 
faire s'enchaîner au mieux.\\

La base de données utilisée pour le logiciel est tirée de la Million Song 
Dataset (MSD). Comme son nom l'indique, c'est une base de données d'un 
million de morceaux, qui contient toutes les informations 
nécessaires sur les pistes, et dont la fiabilité est assurée.
