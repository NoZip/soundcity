
\section{Architecture des modules}
\label{archi:modules}

\subsection{Interface Utilisateur}
\label{archi:modules:iu}

L'Interface Utilisateur (UserInterface) se charge de faire le lien entre
l'utilisateur et le programme. Sa fonction principale est de lancer une
génération de playlist en appelant la fonction \texttt{generate} du module de 
génération. Lors de l'appel à cette fonction, les informations rentrées par 
l'utilisateur sont transmises en paramètres. Lors du lancement du programme,
 une fonction d'initialisation est lancée. Cette dernière permet d'assurer 
 la présence de la base de données ainsi que du, ou des, module(s) de sortie.

\subsection{Générateur}
\label{archi:modules:generateur}

Le module de génération (lui-même composé de 3 sous-modules: Generator, 
SimilarityEngine et FeedbackEngine) s'occupe de la création de la playlist. 
L'interface utilisateur appelle toutes les fonctions du sous-module 
Generator, et celui-ci appelle à son tour toutes les fonctions des deux 
autres sous-modules. La fonction \texttt{initialization} sert à vérifier la 
présence de base de données lors du démarrage du programme. 

La fonction \texttt{generate} sert à générer la playlist d'après les options 
passées via l'interface utilisateur. Elle effectuera un appel à la fonction 
\texttt{select} du module de données afin de récupérer un ensemble fini de morceaux.
 La fonction \texttt{regenerate} est utilisée dans la situation particulière où 
 l'utilisateur se voit proposer de faire une nouvelle génération dans le cas 
 où la génération précédente est jugée non satisfaisante (c.f. Section
  \ref{besoins:fonc}). Cette fonction prend donc les options en paramètres 
 et récupère la précèdente playlist générée dans le but de l'améliorer.

Le sous-module SimilarityEngine sert à calculer la similarité entre deux 
morceaux afin de vérifier la cohérence et la validité de la playlist. Le 
sous-module FeedbackEngine permet de communiquer avec l'interface 
utilisateur pour apporter un visuel de la progression de la génération afin 
d'informer l'utilisateur de l'avancement de la création, et l'aider à 
patienter.

\subsection{Module de données}
\label{archi:modules:donnees}

Le module de données (SQLiteDatabase) a pour objectif d'assurer la 
communication entre le module de génération et la base de données utilisée. 
Sa principale fonctionnalité est d'effectuer une séléction dans la base de 
données en fonction du nombre de chansons souhaité et des paramètres de 
séléction entrés par l'utilisateur. Cette opération se fait grâce à la 
fonction \texttt{select}. De plus, lors du lancement du programme, une 
fonction d'initialisation est appellée afin de certifier la présence de la 
base de données.

\subsection{Module de sortie}
\label{archi:modules:sortie}

Le module de sortie (ici TextOutput) est en charge de fournir à 
l'utilisateur un moyen de récuperer le résultat de la génération demandée. 
Ce module (ainsi que tous les modules de sortie pouvant être crées) devront 
respecter l'interface OutputStrategy pour pouvoir être utilisés sur le 
programme. Dans le cas du module TextOutput, le résultat obtenu sera mis 
sous la forme d'un fichier texte contenant les titres des différents 
morceaux de la playlist. Cette fontionnalité est assurée par la fonction 
\texttt{format} à laquelle on passe la playlist générée en paramètre.

\section{Structures de données}
\label{archi:structures}

\subsection{Morceau}
\label{archi:structures:morceau}

Un morceau (Track) est la structure principale du programme. Cette dernière 
peut être utilisée tant pour décrire un morceau que pour décrire des 
paramètres (certains paramètres seront utilisés uniquement dans le premier 
cas). Les paramètres sont triés dans deux sous-sections (signal\_data et 
context\_data) dans un soucis de visibilité, à l'exception d'un \textit{id} 
qui concerne l'identifiant dans la base de données du morceau (utilisé pour 
éviter des duplications).

\subsubsection{Données signal}
\label{archi:structures:morceau:signal}

Les données signal, représentées par la structure SignalData, sont 
l'ensemble des valeurs correspondant aux déscripteurs audio du morceau, à 
savoir: le rythme, l'énergie et la tonalité (\textit{rythme}, \textit{energy}
 et \textit{tonality}). Ces valeurs sont exprimées par des nombres flottants 
 compris entre 0 et 1.

\subsubsection{Données de contexte}
\label{archi:structures:morceau:contexte}

Les données de contexte sont représentées par la structure ContextData, qui 
contient les informations sur le morceau qui ne sont pas relatives au 
signal : elles décrivent le contexte culturel du morceau. Ces informations 
sont :  un score de popularité exprimé par un nombre flottant compris entre 
0 et 1, l'artiste qui est à l'origine du morceau et l'album sur lequel il 
se trouve.

\subsubsection{Informations sur l'artiste et l'album}
\label{archi:structures:morceau:artiste}

Les informations sur l'artiste (Artist) contiennent un identifiant (qui 
dépendra de la base de données), le nom de l'artiste interprète du morceau 
qui sera stocké dans une chaîne de caractères, ainsi qu'une liste d'artistes 
similaires.

L'album contient lui aussi un identifiant, son nom (qui est utilisé par le 
module de sortie dans le cas où c'est nécessaire), l'artiste de l'album 
(pouvant différer de celui du morceau dans le champ \textit{artist}, par 
exemple dans le cas d'une collaboration avec une autre personne), et l'année 
de sortie.

Ces deux sections appartienent aux données de contexte.

\subsection{Playlist et échantillon de morceaux}
\label{archi:structures:playlist}

La Playlist et l'échantillon de morceaux sont deux listes de morceaux. La 
première est remplie par le module de données, et la deuxième par le 
générateur. La Playlist contient un booléen qui permet de renseigner la 
satisfaction ou non du résultat à l'interface utilisateur, afin de proposer, 
dans le cas où la playlist n'est pas satisfaisante, une nouvelle génération 
à l'utilisateur.

\subsection{Paramètres}
\label{archi:structures:param}

Les paramètres du programme entrés par l'utilisateur seront représentés et 
stockés sous la forme d'un morceau. Les champs non-utilisés par le programme 
seront ignorés, et les champs non-renseignés par l'utilisateur seront 
affectés à des valeurs spécifiques afin de signifier qu'elles ne sont pas 
utilisées : les champs de type \textit{string} seront vides, les champs de 
type \textit{float} auront la valeur 2 (puisque nous n'utilisons que des 
valeurs comprises entre 0 et 1), et les années de sortie de type 
\textit{int} seront mises à 0.