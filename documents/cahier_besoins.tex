\documentclass{article}

\usepackage[utf8]{inputenc}
\usepackage[T1]{fontenc}
\usepackage[francais]{babel}

\title{Cahier des Besoins}
\author{Lucas Vauzelle \and Baptiste Fumaroli \and Jean Pecqueur \and Thomas Pizon}
\date{\today}

\begin{document}

\maketitle

\part{Besoins Fonctionnels}

Ce projet est séparé en plusieurs modules indépendants qui communiquent entre
eux~:
\begin{itemize}
\item Module de génération
\item Module de données
\item Module de sortie
\item Module d'interface utilisateur
\end{itemize}

\section{Module de génération}

Ce module doit générer une playlist de morceaux cohérente et répondant aux
données envoyées depuis l’interface utilisateur. Cet étapes se décompose en
deux besoins différents: la sélection et l'ordonnancement. Après ces étapes,
la playlist doit être communiquer au module de sortie, en fonction de la
solution de sortie choisie par l’utilisateur.

\subsection{Sélection}

Durant la phase de sélection, le générateur doit se connecter au module de
base de données pour récupérer les morceaux correspondants aux facteurs
suivants~:
\begin{itemize}
\item La durée de la playlist ou le nombre de morceaux désirés.
\item Un style (optionnel), choisi dans un ensemble de styles prédéfinis.
\item Un artiste (optionnel).
\item Une intervalle d'années.
\item La dancabilité des morceaux (optionnel, en pourcentage).
\item La popularité minimale des morceaux (optionnel, en pourcentage).
\item Entre 3 et 5 paramètres définissant un profil de morceaux (optionnel,
pourcentages, a définir)
\end{itemize}

\subsection{Ordonnancement}

Ensuite dans cette deuxième phase, les morceaux sont ordonnées pour former une
liste cohérente mais variée de morceaux. Pour cela il faut calculer la
similarité (score en pourcentage) des morceaux entre eux. Un enchaînement
entre deux morceaux trop différents (seuil à déterminer) provoquera une perte
de cohérence. Et un enchaînement entre deux morceaux trop proches (seuil à
déterminer) provoquera une perte de variété. Les doublons de morceaux doivent
être évités.

\section{Module de données}

Ce module devra répondre aux requêtes du module de génération et produire une
sortie utilisable par ce module. Il ne récupérera qu’une sous partie des
entrées correspondant aux critères pour éviter une perte de performances lors
de l'ordonnancement (env 1000).

\section{Module de sortie}

Ce module devra être capable de produire une représentation de la playlist
utilisable par d’autres programmes. Par exemple un affichage sous forme de
texte.

Modules de sortie envisagés~:
\begin{itemize}
\item Texte.
\item Playlist Deezer (optionnel).
\end{itemize}

\section{Module d'interface utilisateur}

Ce module permettra à l’utilisateur de rentrer les données nécéssaires au
module de génération de playlist.

Modules d'interface envisagés~:
\begin{itemize}
\item Console.
\item Interface graphique (optionnel).
\item Interface web (optionnel).
\end{itemize}

\part{Besoins non-fonctionnels}

\begin{itemize}
\item La sélection et l'ordonnancement de morceaux doivent être suffisamment
rapides (15s maximum)
\item Les modules de sortie, de base de données, et d’interface doivent
pouvoir être changés sans altérer le fonctionnement du module de génération.
\item Interface simple et intuitive, destiné à un large public (optionnel).
\item Cohérence globale de la playlist.
\end{itemize}

\end{document}